
%Breite der Bild/Tabellenbeschriftungen:
%\setlength{\LTcapwidth}{\captionwidth}

%Schriftgr��e/Zeilenvorschub in Tabellen:
\def\tablefontsize{\normalsize}
\def\tablespacing{1.0} % linespacing in tables
  
\newcommand{\bc}{\begin{center}}
\newcommand{\ec}{\end{center}} 

\newcommand{\be}{\begin{equation}}
\newcommand{\ee}{\end{equation}} 
\newcommand{\bea}{\begin{eqnarray}}
\newcommand{\eea}{\end{eqnarray}}

\newcommand{\bi}{\begin{itemize}}
\newcommand{\ei}{\end{itemize}}

\newcommand{\bmp}{\begin{minipage}[c]{\columnwidth} \bc}
% argument: vertical position: c, t, b
\newcommand{\emp}{\ec \end{minipage} }

\newcommand{\beg}{\hspace{0.5cm}\begin{minipage}{14cm}}
\newcommand{\eeg}{\end{minipage}\vspace{1cm}}


%%%%%%%%%%%%%% tables %%%%%%%%%%%%%%%%%%%%%%%
\newcommand{\bt}[1]
% #1: table placing: htbp
    {    \begin{table}[#1] %htbp 
         \begin{center}
         \renewcommand{\baselinestretch}{\tablespacing}
         \tablefontsize
    }

%Tabelle mit gr��erem Zeilenvorschub:
\newcommand{\btline}[2]
% #1: table placing: htbp
% #2: linespacing in table, e.g. 1.4
{   \def\tablespacing{#2}
    \bt{#1}
} 

\newcommand {\et}[2] 
% #1: caption
% #2: label without 'tab:'
    {
     \caption{#1.} 
     \label{tab:#2}
     \end{center} 
     \end{table}
     \def\tablespacing{1.0} 
    % reset linespacing for next table
}

\newcommand {\btab}{\begin{tabular}}
\newcommand {\etab} {\end{tabular}}

\newcommand{\mc}{\multicolumn}
\newcommand{\mr}{\multirow}

% columns in math mode:
\newcolumntype{C}{>{$}c<{$}}
\newcolumntype{L}{>{$}l<{$}}
\newcolumntype{R}{>{$}r<{$}}

\newcommand{\pbs}[1]{\let\temp=\\#1\let\\=\temp}
%preserve backslash, s. companion, p108



%%%%%%%%%%%%%% figures %%%%%%%%%%%%%%%%%%%%%%%
\newcommand{\fig}[4]{
% Bild mit 80% der Seitenbreite
% arguments:	
% #1: file without extension eps
% #2: caption
% #3: label
% #4: placing of the figure: e.g. htbp
\begin{figure}[#4]			      
\begin{center}
   \includegraphics[width=.8\columnwidth]{figures/#1}
   \caption{#2.}
   \label{fig:#3}
\end{center}
\end{figure}
}

\newcommand{\figscale}[5]{
% arguments:	
% #1: file without extension eps
% #2: caption
% #3: label
% #4: placing of the figure: e.g. htbp
% #5: width of figure
\begin{figure}[#4]			      
\begin{center}
   \includegraphics[width=#5\columnwidth]{figures/#1}
   \caption{#2.}
   \label{fig:#3}
\end{center}
\end{figure}
}

\newcommand{\figscaletwo}[6]{
% arguments:	
% #1: file1 without extension eps
% #2: file2 without extension eps
% #3: caption
% #4: label
% #5: placing of the figure: e.g. htbp
% #6: width of figure
\begin{figure}[#5]			      
\begin{center}
   \includegraphics[width=#6\columnwidth]{figures/#1}~\\
   \includegraphics[width=#6\columnwidth]{figures/#2}
   \caption{#3.}
   \label{fig:#4}
\end{center}
\end{figure}
}

\newcommand{\figtwo}[5]{
% two figures labeled (a) and (b) side by side
% arguments:	
% #1: file1 without extension eps
% #2: file2 without extension eps
% #3: caption without fullstop
% #4: label
% #5: placing of the figure: e.g. htbp
\begin{figure}[#5]			      
\begin{center}
    \btab{cc} 
   \includegraphics[width=0.45\columnwidth]{figures/#1}
    &	 
   \includegraphics[width=0.45\columnwidth]{figures/#2}
    (a)&(b)
    \etab 
	\caption{#3.}
	\label{fig:#4}
\end{center}
\end{figure}
}

\newcommand{\figTwo}[6]{
% two figures labeled (a) and (b) on top/bottom
% arguments:	
% #1: file1 without extension eps
% #2: file2 without extension eps
% #3: caption without fullstop
% #4: label
% #5: placing of the figure: e.g. htbp
% #6: width of figures as ratio, e.g. 0.7
\begin{figure}[#5]			      
\begin{center}
   \includegraphics[width=#6\columnwidth]{figures/#1}
(a)\\
\vspace{1 em}
   \includegraphics[width=#6\columnwidth]{figures/#2}
    (b)\\
    \caption{#3.}
    \label{fig:#4}
\end{center}
\end{figure}
}

%%%%%%%%%%%%%%%%%%%%%%%%%%%%%%%%%%%%%%%%%%%%%%

\newcommand{\C}[1]{\texttt{#1}} %Typewriter-Font
\newcommand{\tild}{\~~\hspace{-1 ex}} %Tilde ~

\newcommand{\equi}{\Leftrightarrow} %�quivalenz <=>
\newcommand{\concl}{\Rightarrow}    %daraus folgt =>
%Matrix:
\newcommand{\matr}[2]{\left ( \begin{array}{#1} #2 \end{array} \right )}
% #1 alignment of colums, e.g. ccc
% #2 contents of matrix
%Vektor:
\newcommand{\vect}[1]{\left ( \begin{array}{c} #1 \end{array} \right )}
%Fettschrift im Mathemodus:
\newcommand{\mbf}[1]{\mathbf{#1}} %boldface
\newcommand{\gbf}[1]{\boldsymbol{#1}} %boldface greek letters
%Verweis auf Gleichungen mit Klammern um Gl.-Nr.
%\newcommand{\eqref}[1]{(\ref{#1})}
% \eqref defined in package amsmath


\newcommand{\freqorig}{\ensuremath{\circ\hspace{-.1em}\mbox{---}\!\bullet}}
\newcommand{\freq}{\mbox{~}{\circ\hspace{-.1em}\mbox{---}\!\bullet}\mbox{~}}
	% correspondence symbol time<->freq 
\newcommand{\freqswap}{\ensuremath{\bullet\!\mbox{---}\hspace{-.1em}\circ}}
\newcommand{\freqv}{\begin{turn}{90} \freqswap \end{turn}}
	% vertical correspondence symbol
	% time 
	% freq
\newcommand{\timeh}{\mbox{~}{\bullet\!\mbox{---}\hspace{-.1em}\circ}\mbox{~}}
	% correspondence symbol freq<->time
\newcommand{\timev}{\begin{turn}{90} \freqorig \end{turn}}
	% vertical correspondence symbol
	% freq
	% time

\newcommand{\Case}[1]{\left\{ \begin{array}{ll} #1
    \end{array}\right. } 


\newcommand{\Mbox}[1]{
\ifthenelse{\equal{\boxes}{yes}}
{%\ensuremath
\fbox{$\displaystyle #1$}}
{#1}}

%box around a equationarray
%argument is the contents of the eqnarray environment
\newcommand{\Mboxarray}[1]{
\ifthenelse{\equal{\boxes}{yes}}
{\bc\fbox{\begin{Beqnarray} #1 \end{Beqnarray}}\ec}
{\bea #1 \eea}
}


\newcommand{\suml}{\sum\limits}
% limits of sum UNDER the sign instead of beside it, e.g. in fractions

\newcommand{\intl}{\int\limits}
% limits of int UNDER the sign instead of beside it


\newcommand{\logd}{\mbox{ld\,}} %base 2 logarithm

\newcommand{\erw}[1]{E\left\{#1\right\}} %Erwartungswert
