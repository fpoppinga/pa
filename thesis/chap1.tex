\chapter{Erstes Kapitel}
\ifenglish
bla bla bla bla bla bla bla bla bla bla bla bla bla bla bla
``bla bla bla'' bla bla bla bla bla bla bla bla bla bla bla bla
bla bla bla bla bla bla bla bla bla bla bla bla bla bla bla
\else
bla bla bla bla bla bla bla bla bla bla bla bla bla bla bla
"`bla bla bla"' bla bla bla bla bla bla bla bla bla bla bla bla
\glqq bla bla bla\grqq~bla bla bla bla bla bla bla bla bla bla bla bla
\fi

bla bla bla bla bla bla bla bla bla bla bla bla bla bla bla
bla bla bla bla bla bla bla bla bla bla bla bla bla bla bla
bla bla bla bla bla bla bla bla bla bla bla bla bla bla bla
bla bla bla bla bla bla bla bla bla bla bla bla bla bla bla

\section{Abschnitt eins\index{Abschnitt eins}}
bla bla bla bla bla bla bla bla bla bla bla bla bla bla bla
bla bla bla bla bla bla bla bla bla bla bla bla bla bla bla
bla bla bla bla bla bla bla bla bla bla bla bla bla bla bla
bla bla bla bla bla bla bla bla bla bla bla bla bla bla bla

bla bla bla.
Ein Zitat: \cite{fli:mult} und \cite{Bauch_MAS}

Ein Equationarray:
\bea
\hat X(z)&=&P(z)[E(z)+\hat X(z)]\\
\equi\quad\hat X(z)&=&\frac{P(z)}{1-P(z)}E(z)\\
\Omega &=& 0,25 \pi\\
\mathbf{Ax} &=&\mathbf{a}
\eea

\index{�bertragungstest(1)} 
\index{Ubertragungstest@�bertragungstest(2)} % => Sortierung unter U
\index{"Ubertragungstest(3)} 
	% => Sortierung unter Ue bei Option -g -s index_umlauts.ist 
\index{\"Ubertragungstest(4)} 
\index{Ubers}
\index{Uebers}

Eine Gleichung
\be
\sum_{k=1}^{m}a^{(m)}_k r_{xx}(i-k)= r_{xx}(i)\quad,\:i=1,\ldots,m
\label{eq:normalgl}
\ee
und mit Box:
\be
\Mbox{
\sum_{k=1}^{m}a^{(m)}_k r_{xx}(i-k)= r_{xx}(i)\quad,\:i=1,\ldots,m
}.
\label{eq:normalglbox}
\ee

Gleichung \eqref{eq:normalglbox} hat eine Box, Gleichung \eqref{eq:normalgl} nicht!

Hier ein Verweis auf Bild \vref{fig:bildlabel}.

Hinweis: Hier wird ein Makro zur Einbindung des Bildes benutzt (siehe file 
\texttt{macros.tex}).\index{Bild einbinden!mit Makro}

\figscale{hcos}{Bildunterschrift bla bla bla bla bla bla bla bla bla bla bla bla bla bla bla bla bla bla bla bla bla bla bla bla bla bla bla bla bla bla}{bildlabel}{ht}{.7}
% Hinweis: Skalierung (.7) ist bezogen auf nutzbare Seitenbreite

Und auf die Tabelle \vref{tab:res2wav}. 

\btline{ht}{1.2}
\btab{|c|l|c|c|c|}
 \hline
Nr.& Datei& Dauer in s& \mc{1}{c|}{$G_p/$ dB}& \mc{1}{c|}{$\mbox{seg}G_p/$ dB}\\
 \hline
 \hline
\mc{5}{|l|}{\sf kurze Sinust�ne 14~kHz bis 20~kHz in Halbtonschritten}\\
 \hline
1& \C{38\_2\_24.wav}& 20.06& 29.48& 26.52\\
  \hline
\etab
\et{Liste der wav-Dateien mit Pr�diktionsergebnissen}{res2wav}

\section{Abschnitt zwei}

bla bla bla bla bla bla bla bla bla bla bla bla bla bla bla
bla bla bla bla bla bla bla bla bla bla bla bla bla bla bla
bla bla bla bla bla bla bla bla bla bla bla bla bla bla bla

bla bla bla bla bla bla bla bla bla bla bla bla bla bla bla
bla bla bla bla bla bla bla bla bla bla bla bla bla bla bla
bla bla bla bla bla bla bla bla bla bla bla bla bla bla bla
bla bla bla bla bla bla bla bla bla bla bla bla bla bla bla
bla bla bla bla bla bla bla bla bla bla bla bla bla bla bla
bla bla bla bla bla bla bla bla bla bla bla bla bla bla bla
bla bla bla bla bla bla bla bla bla bla bla bla bla bla bla

bla bla bla bla bla bla bla bla bla bla bla bla bla bla bla
bla bla bla bla bla bla bla bla bla bla bla bla bla bla bla
bla bla bla bla bla bla bla bla bla bla bla bla bla bla bla
bla bla bla bla bla bla bla bla bla bla bla bla bla bla bla
bla bla bla bla bla bla bla bla bla bla bla bla bla bla bla
bla bla bla bla bla bla bla bla bla bla bla bla bla bla bla
bla bla bla bla bla bla bla bla bla bla bla bla bla bla bla

bla bla bla bla bla bla bla bla bla bla bla bla bla bla bla
bla bla bla bla bla bla bla bla bla bla bla bla bla bla bla
bla bla bla bla bla bla bla bla bla bla bla bla bla bla bla
bla bla bla bla bla bla bla bla bla bla bla bla bla bla bla

Listing \ref{lst:bsp} zeigt ein Beispiel f�r das Einbinden einer Datei.
\lstinputlisting[float=htp,
								caption={Beispiel-Listing in eigener Umgebung.}, 
								label=lst:bsp,
								numbers=left, stepnumber=1,
									%firstline=7, lastline=38, 
									stepnumber=1]{sourcecodes/fftdb.m}

Variable \lstinline!e_buff! enth�lt... 

Funktion \lstinline!function fftdb! macht dies und das und jenes


\section{Subfigures}

Beispiel f�r Subfigures

\begin{figure}[ht]
	\centering
	\subfigure[Dummy 1]{
		\includegraphics[width=0.35\columnwidth]{figures/dummy}
		\label{fig:dummy1}
	}
	\qquad
	\subfigure[Dummy 2]{
		\includegraphics[width=0.35\columnwidth]{figures/dummy}
		\label{fig:dummy2}
	}
	\caption{Zwei Dummies in einer Figure}
	\label{fig:dummy12}
\end{figure}

Bild \ref{fig:dummy1} und \ref{fig:dummy1} sind in \ref{fig:dummy12} zu sehen.